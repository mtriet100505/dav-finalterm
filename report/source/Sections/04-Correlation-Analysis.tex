\section{Correlation Analysis}

In this task, we will analyze the correlations between multiple variables/features in our dataset. There are two correlation coefficients we can utilize: Pearson and Spearman, but we will focus only on Spearman, which will be discussed later in Section~\ref{subsec:spearman}.

\subsection{Theoretical Basis of Spearman Correlation}
\label{subsec:spearman}

The \textit{Spearman rank correlation coefficient} ($\rho$) measures the statistical dependence between two variables. Unlike the \textit{Pearson correlation coefficient} $r$, it does not assume linearity or a normal distribution. Instead, it evaluates whether the relationship between two variables is monotonic---that is, as one variable increases, the other consistently increases or decreases, though not necessarily at a constant rate. 

$\rho$ works especially well for \textbf{ordinal categorical} variables and is robust to outliers, since it relies on ranks rather than raw values.

Assuming a dataset with a sample size of $n$, $\rho$ is computed by ranking the values of each variable:

$$\rho \in [-1, 1] = 1 - \frac{6 \times \text{sumsq}(R(X_i)-R(Y_i))}{n(n^2-1)}$$

where:

\begin{itemize}
    \item $\text{sumsq}(\cdot)$: sum of squares of its elements.
    \item $R(\cdot)$: the rank of a given variable value.
    \item $X_i, Y_i$: the value of two variables $X$ and $Y$ of the $i^{th}$ observation.
\end{itemize}

The meaning of $\rho$ is as follows:

\begin{itemize}
    \item $\rho = 1$: perfect positive monotonic relationship (upward trend).
    \item $\rho = -1$: perfect negative monotonic relationship (downward trend).
    \item $\rho = 0$: no monotonic relationship (random fluctuations).
\end{itemize}

\subsection{Experiment Results}

Our target variable of choice for this task is salary (USD), and we are going to compare it with four different feature variables: work year, remote ratio, experience level, and company size.

Since both experience level and company size's data types are categorical strings, we need to encode them in the correct ordinal order.

\begin{figure}[!htpb]
    \centering
    \includegraphics[width=\linewidth]{Figures/OrdinalEncode.png}
    \caption{Ordinal encoding for categorical variables, showing the last five entries, with before and after comparison.}
    \label{fig:ordinal-encode}
\end{figure}

We can begin calculating $\rho$ for each combinations of feature and target variables. For this, we use Python's \texttt{scipy.stats.spearmanr()}, which conveniently returns both $\rho$ and $p$ (the probability of seeing our observed statistic, as mentioned in Section~\ref{subsec:anova}). The results are then shown in Figure~\ref{fig:spearman}.

\begin{figure}[!htpb]
    \centering
    \includegraphics[width=1\linewidth]{Figures/Spearman.png}
    \caption{$\rho$ and $p$ values from performing Spearman correlation analysis using \texttt{scipy.stats.spearmanr()}.}
    \label{fig:spearman}
\end{figure}

All variables have $p < \alpha = 0.05$, meaning the observations we observe are statistically significant rather than random. In other words, each feature shows a meaningful monotonic relationship with salary instead of arising from noise in the data.

Furthermore, experience level has the highest positive $\rho$, which means it has a plausible and stable upward trend with salary; meanwhile other variables have $\rho \approx 0$, corresponding unstable fluctuations, especially for remote work which matches our previous observations that hybrid work is significantly lower.

\subsection{Plotting Visualizations}

A correlation heatmap (Figure~\ref{fig:correlation-heatmap}) is useful to display all $\rho$ values of variable pairs in a single matrix.

\begin{figure}[!htpb]
    \centering
    \includegraphics[width=\linewidth]{Figures/CorrHeatmap.png}
    \caption{Correlation heatmap of $\rho$ between all variable pairs.}
    \label{fig:correlation-heatmap}
\end{figure}

While our main goal is to examine the monotonic relationship with salary, it is worth noting that both remote ratio and work year show the highest negative $\rho$. This further supports our earlier explanations regarding the bias toward increased remote work in recent years.

To better visualize monotonicity, we revisit the boxplots, this time adding regression lines to show trends or fluctuations (Figure~\ref{fig:boxplot-regression}).

\begin{figure}[!htpb]
    \centering
    \includegraphics[width=\linewidth]{Figures/BoxplotRegression.png}
    \caption{Boxplots of all features against salary, with dashed regression lines representing trends and monotonicity.}
    \label{fig:boxplot-regression}
\end{figure}

All boxplots match our observation that $p$ values are extremely small, indicating that the detected trends are meaningful rather than random noise.

Key insights:

\begin{enumerate}
    \item Experience level shows a roughly linear upward trend, consistent with having the highest $\rho$ among the four variables tested.
    \item The regression lines of both remote ratio and company size form a unique V shape, aligning with its $\rho$ being close to 0.
    \item Work year also shows an upward trend, though not consistently.
\end{enumerate}

\subsection{Observations \& Conclusions}

Our statistical results reflect practical real-world patterns: salary increases as employees gain more experience during their work, and companies generally prefer hiring more experienced employees for higher-responsibility roles, naturally resulting in higher pay; remote work has become increasingly relevant and serves as a flexible, viable working model; work year correlates with salary due to skill growth and inflation; and company size tends to scale with salary, but visualizations indicate that the change from M to L is downward.

\textbf{Conclusion:} Our correlation analysis shows that salary has monotonic relationships with work year, and experience level, while remote ratio and company size show little to no monotonicity.