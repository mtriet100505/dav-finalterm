% Conclusion
\section{Summary}

This report presents a comprehensive analysis of salary data, exploring the factors that influence compensation in the context of experience, remote work, and organizational characteristics. The following key insights summarize our findings:

\begin{enumerate}
    \item Experience level strongly correlates with salary. Employees at higher levels such as Senior and Executive consistently earn more than those at Entry or Mid levels. Statistical analysis, including Spearman correlation and multiple linear regression, confirms a stable upward trend in salary with increasing experience.
    \item The rise of remote work has introduced nuanced effects on salaries. ANOVA results show significant differences across remote ratios. Fully remote and on-site roles exhibit similar median salaries, both generally higher than hybrid positions. This suggests that while remote work has become more prevalent, its impact on pay varies depending on work structure.
    \item Additional variables such as company size, work year, and job title also contribute to salary differences. Company size shows a moderate association, with larger organizations typically offering higher compensation, although some inconsistencies exist. Work year correlates positively with salary, reflecting experience growth and inflation trends.
\end{enumerate}

Overall, the analysis indicates that salary is primarily driven by experience level, with remote work and organizational factors also playing significant roles. Recognizing these relationships allows for more informed decision-making regarding compensation, career planning, and organizational policies.